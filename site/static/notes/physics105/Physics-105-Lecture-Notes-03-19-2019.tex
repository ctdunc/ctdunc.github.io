\documentclass{article} \usepackage{amsmath} \usepackage{amssymb} \usepackage{amsthm} \usepackage[margin=0.2in]{geometry} \usepackage{hyperref} \usepackage{physics} \usepackage{tikz} \usepackage{mathtools} \mathtoolsset{showonlyrefs} \theoremstyle{definition} \newtheorem{theorem}{Theorem}[section] \newtheorem{corollary}{Corollary}[theorem] \newtheorem{lemma}[theorem]{Lemma} \newtheorem{definition}{Definition}[section] \author{Connor Duncan} \date{\today}
\title{Physics-105-Lecture-Notes-03-19-2019}
\begin{document}
\maketitle\tableofcontents
\noindent\abstract{A single PDF with all lectures in a single document can be downloaded at \url{https://www.dropbox.com/sh/8sqzvxghvbjifco/AAC9LoSRnsRQDp7pYedgWpQMa?dl=0}. The password is 'analytic.mech.dsp'.
 This file was automatically generated using a script, so there might be some errors. If there are, you can contact me at \url{mailto:ctdunc@berkeley.edu}.}
\subsection{Inertial Tensor} \begin{align} \Omega_1=\sqrt{\left(\frac{I_2-I_1}{I_2}\right)\left(\frac{I_2-I_1}{I_1}\right)}\omega_2\\ \end{align} cyclic differential equation \begin{align} \dot\omega_1+\left(\frac{I_3-I_2}{I_1}\right)\omega_2\omega_3=0\\ \dot\omega_2+\left(\frac{I_1-I_3}{I_2}\right)\omega_1\omega_3=0\\ \dot\omega_3+\left(\frac{I_2-I_1}{I_3}\right)\omega_2\omega_1=0 \end{align} \subsection{Spinning Top} If we want to be more rigorous, we should include gravity and torque. Let some coordinate system $\vec{r}'\equiv$ space coordinates $\equiv$ fixed in space. $\vec{r}$ is our body coordinates, rotating with some spinning top. These are related by $\vec{r}'=U\vec{r}$. Also $\theta,\varphi,\psi$ are the euler angles. We need to choose a convenient definition for these, so let $\theta$ be the angle between $z,\hat x_3'$, and $\varphi$ the angle betwween $x,\hat x_2'$, and $\psi$ the angle between the $xy$ plane and $\hat x_2'$. So, step 1 is that if we rotate around $z-\hat x_3$ by some angle $\varphi$ (in $xy$ plane), we get new coordinate transformation $\alpha,\beta\gamma$, which gives \begin{equation} \begin{bmatrix} \alpha\\\beta\\\gamma \end{bmatrix} = \begin{bmatrix} \cos\varphi & -\sin\varphi & 0\\ -\sin\varphi & \cos\varphi & 0\\ 0 & 0 & 1 \end{bmatrix} \begin{bmatrix} x\\y\\z \end{bmatrix} \end{equation} \subsubsection{Step 2} Rotate around $\alpha$ by angle $\theta$. Su, we get \begin{equation} \begin{bmatrix} \alpha'\\\beta'\\\gamma' \end{bmatrix} =\begin{bmatrix} 1 & 0 & 0\\ 0 & \cos\theta & \sin\theta\\ 0 & -\sin\theta & \cos\theta \end{bmatrix} \begin{bmatrix} \alpha\\\beta\\\gamma \end{bmatrix} \end{equation} \subsubsection{Step 3} Finally, wrotate by $\psi$ about $\gamma'$ \begin{equation} \begin{bmatrix}x_1\\x_2\\x_3\end{bmatrix} = \begin{bmatrix} \cos\psi & \sin\psi & 0\\ -\sin\psi & \cos\psi & 0\\ 0 & 0 & 1 \end{bmatrix} \end{equation} \subsection{gettting the euler angles! (space 2 body)} $U^*=U_3^*U_2^*U_1^*$ transforms from space to body, with $U=U_1U_2U_3$ transforms from body to space. We get the big transpose form of $U$ as \begin{equation} U^*= \begin{bmatrix} \cos\psi\cos\varphi-\sin\psi\sin\varphi\cos\theta & \cos\psi\sin\varphi\sin\psi\cos\varphi\cos\theta & \sin\theta\sin\psi\\ -\sin\psi\cos\varphi-\cos\psi\sin\varphi\cos\theta & -\sin\varphi\sin\psi+\cos\psi\cos\varphi\cos\theta & \sin\theta\cos\psi\\ \sin\theta\sin\varphi & -\cos\varphi\sin\theta & \cos\theta \end{bmatrix} \end{equation} <++>
\end{document}
