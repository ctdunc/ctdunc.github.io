\documentclass{article} 
\usepackage{amsmath} 
\usepackage{amssymb} 
\usepackage{amsthm} 
\usepackage[margin=0.2in]{geometry} 
\usepackage{hyperref} 
\usepackage{physics} 
\usepackage{tikz} 
\usepackage{mathtools}
\mathtoolsset{showonlyrefs} 
\theoremstyle{definition} 
\newtheorem{theorem}{Theorem}[section] 
\newtheorem{corollary}{Corollary}[theorem] 
\newtheorem{lemma}[theorem]{Lemma} 
\newtheorem{definition}{Definition}[section] 

\author{Connor Duncan}
\date{\today}

\title{notes-10-02-2019}
\begin{document}
\abstract{A single document copy of these notes, as well as a mirror of every note, can be found at \url{connorduncan.xyz/notes}}
\subsection{Identical Fermions with Spin} Consider $\psi(x_1,x_y,\sigma_1,\sigma_2)$. If no spin-orbit coupling, we just get $\psi(x_1,x_2)\chi(\sigma_1,\sigma_2)$. The product of the orbital DOF, and the spin DOF. Without tying ourselves to a basis we obtain $\ket{\psi}\otimes\ket{\chi}$, but this is kind of trivial. The important thing is that our wf must be antisymmetric wrt exchange. There are two possibilities \begin{itemize} \item $\psi_A(x_1,x_2)\chi_S(\sigma_1,\sigma_2)$. \item $\psi_S(x_1,x_2)\chi_A(\sigma_1,\sigma_2)$. \end{itemize} When we're actually adding our spin multiplet, we get that (and you will recall) \begin{itemize} \item $s=0$ corresponds to $\frac{1}{\sqrt{2}}(\ket{\uparrow\downarrow}-\ket{\downarrow\uparrow})$ \item $s=1$ corresponds to $\frac{1}{\sqrt{2}}(\ket{\uparrow\downarrow}+\ket{\downarrow\uparrow}),\ket{\uparrow\uparrow},\ket{\downarrow,\downarrow}$. \end{itemize} These correspond to the functions \begin{align} \chi_S(\sigma_1,\sigma_2)=\left\{ \begin{matrix*}[l] \chi & \sigma_1 & \sigma_2\\\hline \frac{1}{\sqrt{2}} & \uparrow & \downarrow\\ \frac{1}{\sqrt{2}} & \downarrow & \uparrow\\ 0 & \uparrow & \uparrow\\ 0 & \downarrow & \downarrow \end{matrix*} \right. && \chi_A(\sigma_1,\sigma_2)=\left\{ \begin{matrix*}[l] \chi & \sigma_1 & \sigma_2\\\hline \frac{1}{\sqrt{2}} & \uparrow & \downarrow\\ -\frac{1}{\sqrt{2}} & \downarrow & \uparrow\\ 0 & \uparrow & \uparrow\\ 0 & \downarrow & \downarrow \end{matrix*} \right. \end{align} This basically lets us index our $\chi_{S,S_z}$ by the additional $S_z$ term when the spin state must be symmetric, and only by the spin-0 state when $\chi$ must be antisymmetric. \subsubsection{The Helium Atom} For helium, what is our hamiltonian? We have two particles orbiting a nucleus. We will approximate the mass of the proton to be roughly infinite (otherwise it's really hard), so we'll take \begin{equation} \hat H=\frac{p_1^2}{2m}+\frac{p_2^2}{2m}-\frac{ze^2}{|\vec{r}_1|}-\frac{ze^2}{|\vec{r}_2|}+\frac{e^2}{|\vec{r}_1-\vec{r}_2|} \end{equation} where $z=2$. Our first attempt at this will utilize perturbation theory to solve it, by treating the interaction term as a perturbation. \paragraph{Perturbative Approach} In the perturbation scheme, we can think of \begin{align} H_0=\frac{p_1^2}{2m}+\frac{p_2^2}{2m}-\frac{ze^2}{r_1}-\frac{ze^2}{r_2} && H_1=\frac{e^2}{|r_1-r_2|} \end{align} with \begin{equation} \hat H=H_0+H_1 \end{equation} we now take the 0th order in perturbation theory, and basically think of it as the hydrogen atom, but coupled. We should note this is a pretty bad approach. Since we have that $z=2$, $H_1\sim H_0$, so we're going to get qualitatively correct but quantitatively incorrect results. Note that to 0th order, we have \begin{equation} \psi_0^{(0)}=\tilde\varphi_{100}(r_1)\tilde\varphi_{100}(r_2)\chi_A(r_1,r_2) \end{equation} In general to first order, we can write \begin{equation} \psi^{(0)}=\left[\varphi_{100}(r_1)\varphi_{n\ell m}(r_2)\pm\varphi_{100}(r_2)\varphi_{n\ell m}(r_1)\right] \chi_{A,S}(r_1,r_2) \end{equation} Where the spin wavefunction is antisymmetric when the internal $\pm\mapsto+$, and symmetric when it's $-$. The names of this particle are $s=0$: Parahelium. $s=1$: Orthohelium. Note that we also have degeneracies in energy, regardless of the sign. We have 1 state para, 3 states ortho. These four states are degenerate to first order, but we can lift this degeneracy with the interaction term. This happens because the spin component of the wavefunction is correlated with the orbital state. This turns out to be the source of basically all magnetism we see. Let's do the ground state shift. Recall that for hydrogen (letting $a=\frac{a_0}{z}$, where $a_0$ is the bohr radius $a_0=\frac{\hbar^2}{me^2}$) \begin{equation} \psi_{100}(r)=\sqrt{\frac{1}{\pi a^3}}e^{-r/a} \end{equation} We compute (assuming resolution of identity, etc for position repr) \begin{equation} \Delta E_0=\bra{\psi^{(0)}}H_1\ket{\psi^{(0)}}= \int d^3r_1d^3r_2|\varphi_{100}(r_1)|^2|\varphi_{100}(r_2)|^2\frac{e^2}{|r_1-r_2|} \end{equation} Since the square of the wavefunction is really just the probability of finding it at $r$, we can bring an $e$ inside both of them, and reexpress this as what is essentially an electrostatic calculation, letting $\rho$ be charge density \begin{equation} \int d^3r_1d^3r_2\frac{\rho(r_1)\rho(r_2)}{|r_1-r_2|} \end{equation} We skipped the actual computation of this in lecture, but we can continue, skipping steps to get \begin{equation} \frac{ze^2}{2a_0(4\pi)^2}\int d^3x_1d^3x_2\frac{e^{-(x_1+x_2)}}{|x_1-x_2|} \end{equation} Where \begin{equation} \frac{1}{(4\pi)^2}\int d^3x_1d^3x_2\frac{e^{-(x_1+x_2)}}{|x_1-x_2|}=\frac{5}{4} \end{equation} And, we take that one Rydberg$=\frac{e^2}{2a_0}$, we get \begin{equation} \int d^3r_1d^3r_2|\varphi_{100}(r_1)|^2|\varphi_{100}(r_2)|^2\frac{e^2}{|r_1-r_2|}=z\frac{5}{4}\mathrm{Ry} \end{equation} Yielding a full energy correction \begin{equation} E_0^{(1)}=\left(-2z^2+\frac{5}{4}\right)\mathrm{Ry}=-5.5\mathrm{Ry}=-74.8\mathrm{eV} \end{equation} So, we have now $\psi^{(0)}_{n\ell m\alpha}$, where $\alpha$ is either ortho or para. We also have $E_{n\ell m}^{(0)}=E_{100}^{(0)}+E_{n\ell m}^{(0)}$. Our goal now is to forget about degenerate perturbation theory, and pick some nice basis (which I think we can, since we're nondegenerate on $H_1$?) We computed \begin{equation} \Delta E_{n\ell m\alpha}=\bra{\psi^{(0)}_{n\ell m\alpha}}\frac{e^2}{|r_1-r_2|}\ket{\psi^{(0)}_{n\ell m\alpha}} =\frac{e^2}{2}\int d^3r_1d^3r_2 \frac{ |\tilde\varphi{100}(r_1)\tilde\varphi_{n\ell m}(r_2)\pm\tilde\varphi_{100}(r_2)\tilde\varphi_{n\ell m}(r_1)|^2 }{|r_1-r_2|} =J_{n\ell}\pm K_{n\ell} \end{equation} where \begin{align} J_{n\ell}=e^2\int d^3r_2d^3r_2\frac{|\tilde\varphi_{100}(r_1)|^2|\tilde\varphi_{100}(r_2)|^2}{|r_1-r_2|} \\ K_{n\ell}=e^2\int d^3r_1d^3r_2\frac{\Re[\varphi^*_{100}(r_1)\tilde\varphi_{100}(r_2)\tilde\varphi_{n\ell m}(r_2)^*\tilde\varphi_{n\ell m}(r_1)]}{|r_1-r_2|} \end{align} Altman does not want to compute these integrals. They look yucky, so I don't blame him. Basically, since we have two indistinguishable fermions, we can't ever find them in exactly the same state. So, if we think of exchange in orbital space, for the antisymmetric spin state, we will have a symmetric wave function, and vice versa. \footnote{Homework Toy Model Basically, we are going to take some toy helium atom and solve it completeley with perturbation theory and the variational approach. We're going to take $H=H_1+H_2+u\delta(x_1-x_2)$ Bonus points to the person who finds the nicest variational function. } \paragraph{Variational Approach} WTS we can do way better than perturbation theory. Let's go back to our hamiltonian. \begin{equation} \hat H=\frac{p_1^2}{2m}+\frac{p_2^2}{2m}-\frac{2e^2}{|\vec{r}_1|}-\frac{2e^2}{|\vec{r}_2|}+\frac{e^2}{|\vec{r}_1-\vec{r}_2|} \end{equation} Let's consider the ground state representation in the position basis as a variational function, with $z$ a variational parameter. \begin{equation} \tilde\varphi_{100}(r_1)=\sqrt{\frac{z^3}{\pi a_0^3}}e^{-zr/a_0} \end{equation} We're going to write the hamiltonian now as \begin{equation} H=\frac{p_1^2}{2m}-\frac{ze^2}{r_1}+\frac{p_2^2}{2m}-\frac{ze^2}{r_2}+\frac{(z-2)e^2}{r_1}+\frac{(z-2)e^2}{r_2}+\frac{e^2}{|r_1-r_2|} \end{equation} We consider that $\psi_z(r_1,r_2)=\varphi_{100,z}(r_1)\varphi_{100,z}(r_2)$. Then, we compute \begin{align} \bra{\psi_z}H\ket{\psi_z}=-2z^2\mathrm{Ry}+2I_1+\frac{e^2z^3}{\pi a_0^3}\int d^3r_1d^3r_2\frac{e^{-z(r_1+r_2)}}{|r_1-r_2|}=-8z+2z^2+\frac{5}{4}z \end{align} $I_1$ is a pretty simple integral apparently. \begin{equation} I_1=\bra{\psi_z}\frac{(z-2)e^2}{|r_1|}\ket{\psi_z}=\frac{e^2z^3}{\pi a_0^3}\int d^3r_1\frac{(z-2)}{r_1}e^{-zr_1/a_0}=\frac{e^2z^3}{\pi a_0^3}4\pi\int dr_1r_1e^{-zr_1/a_0}(z-2) \end{equation} or something along those lines, it works out though. Altman wasn't really sure his normalization was right, but we have something close and could probably figure it out Either way, we end up with \begin{equation} \pdv{E}{z}=-8+4z+\frac{5}{4}\Rightarrow z_*=2-\frac{5}{16} \end{equation} $z_*$ minimizes the energy. Then, we can plug this back into the energy, and we end up with \begin{equation} E_0=-77.5\mathrm{eV} \end{equation} We got a 1\% approximation, whereas perturbation theory gave us an answer within roughly 10\%.
\end{document}
