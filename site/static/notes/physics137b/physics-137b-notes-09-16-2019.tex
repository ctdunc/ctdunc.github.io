\documentclass{article} 
\usepackage{amsmath} 
\usepackage{amssymb} 
\usepackage{amsthm} 
\usepackage[margin=0.2in]{geometry} 
\usepackage{hyperref} 
\usepackage{physics} 
\usepackage{tikz} 
\usepackage{mathtools}
\mathtoolsset{showonlyrefs} 
\theoremstyle{definition} 
\newtheorem{theorem}{Theorem}[section] 
\newtheorem{corollary}{Corollary}[theorem] 
\newtheorem{lemma}[theorem]{Lemma} 
\newtheorem{definition}{Definition}[section] 

\author{Connor Duncan}
\date{\today}

\title{notes-09-16-2019}
\begin{document}
\abstract{A single document copy of these notes, as well as a mirror of every note, can be found at \url{connorduncan.xyz/notes}}
\subsection{Guest Discussion while Altman out of town (9/16/19)} \subsubsection{System of Two Spin-1/2 Particles} Discussion of this motivated by desire to analyze hyperfine splitting in the hydrogen atom. We begin with discussion of the hamiltonian \begin{equation} \hat H=\frac{2A}{\hbar^2}S_1\cdot S_2 \end{equation} Hilbert space is given by the tensor product of our two spin-1/2 systems. For two bases, $\ket{\uparrow},\ket{\downarrow}$, we have our new space as $\ket{\uparrow\downarrow},\ket{\uparrow\uparrow},\ket{\downarrow\downarrow},\ket{\downarrow\uparrow}$. We can also write our new hamiltonian as \begin{equation} 2S_1\cdot S_2=S_1^+S_2^-+S_1^-S_2^++2S_1^zS_2^z \end{equation} where \begin{equation} S_{1,2}^\pm=S_{1,2}^x\pm iS_{1,2}^y \end{equation} It can be checked explicitly that $\ket{\uparrow\uparrow},\ket{\downarrow\downarrow}$ are still eigenstates of the hamiltonian, since. We want to know then what \begin{align} \bra{\uparrow\downarrow}S_1^+S_2^-+S_1^-S_2^++2S_1^zS_2^z\ket{\uparrow\downarrow} \end{align} By symmetry arguments, a lot of these coefficients go to zero, so we get $\frac{-A}{2}$ for both oriientations on the diagonal, and $A$ on the off. In total, the logic gives the following hamiltonian \begin{equation} \hat H= \begin{bmatrix} \frac{A}{2} & 0 & 0 & 0 \\ 0 & \frac{-A}{2} & A & 0 \\ 0 & A & \frac{-A}{2} & 0 \\ 0 & 0 & 0 & \frac{A}{2} \end{bmatrix} \end{equation} We can get the eigenvalues, vectors of the central square matrix by taking \begin{align} \text{Spin Triplet} && \frac{1}{\sqrt{2}}(\ket{\uparrow\downarrow}+\ket{\downarrow\uparrow}) && \frac{A}{2}\\ \text{Spin Singlet} && \frac{1}{\sqrt{2}}(\ket{\uparrow\downarrow}-\ket{\downarrow\uparrow}) && \frac{-3A}{2} \end{align} Now, we want to see if its possible to invent some symmetry that commutes with our hamiltonian. It seems like it should be symmetric by rotation about $\hat n$ by an angle $\delta\theta$. We can write down $R(\delta\theta)$, which is generated by $\vec{S}\cdot\vec{n}$, which gives \begin{equation} R(\delta\theta)=e^{-i(\vec{S}\cdot\vec{n})\delta\theta} \end{equation} We can show that $[S\cdot\hat n, H]=0$. What it means to take $S\cdot\hat n$ in our tensor product space is \begin{equation} S\cdot\hat n=S_1\hat n\otimes I+I\otimes S_2\hat n \end{equation} \footnote{Is this basically JCF?} \footnote{TODO: Write out spin operators for this state explicitly} These all fulfill the basic requirements of the spin algebra, \begin{equation} [S_x,S_y]=iS_z \end{equation} and cyclic permutations thereof. Alternately, \begin{equation} [S_\alpha, S_\beta]=i\varepsilon_{\alpha\beta\gamma}S_\gamma \end{equation} There's another operator \begin{align} \hat S^2=S_x^2+S_y^2+S_z^2 && [\hat S^2,H]=0 \end{align} Theres another way to write this as \begin{align} S^2=(S_{1x}+S_{2x})^2+(S_{1y}+S_{2y})^2+(S_{1z}+S_{2z})^2 \\ =S_1^2+S_2^2+2S_1\cdot S_2 \end{align} All of those individually commute with the hamiltonian, since $[S_1^2,S_1\cdot S_2]=0$, and same for $S_2$. We call $S^2$ the \emph{total spin operator}. With $S_{x,y,z}, S^2$ the total spin operators. Now, check the eigenstates of the hamiltonian are eigenstates of the spin operators. \begin{align} \begin{matrix*}[l] & \hat H & \hat S^2 & \hat S_z\\ \hline \ket{\uparrow\uparrow} & A/2 & 2\hbar^2 & \hbar\\ \frac{1}{\sqrt{2}}(\ket{\uparrow\downarrow}+\ket{\downarrow\uparrow}) & -A/2 & 2\hbar^2 & 0\\ \ket{\downarrow\downarrow} & A/2 & 2\hbar^2 & -\hbar\\ \frac{1}{\sqrt{2}}(\ket{\uparrow\downarrow}-\ket{\downarrow\uparrow}) & -A/2 & 0 & 0 \end{matrix*} \end{align} We can see the first 3 are the spin-1 system, and the final system the spin-0. \subsubsection{Problems 11.16, 11.18 in Townsend} \paragraph{11.16} Considering $H_{\text{Hydrogen}}+\frac{\gamma}{r}$.
\end{document}
