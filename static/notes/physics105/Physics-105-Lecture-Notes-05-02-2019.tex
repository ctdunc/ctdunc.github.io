\documentclass{article} \usepackage{amsmath} \usepackage{amssymb} \usepackage{amsthm} \usepackage[margin=0.2in]{geometry} \usepackage{hyperref} \usepackage{physics} \usepackage{tikz} \usepackage{mathtools} \mathtoolsset{showonlyrefs} \theoremstyle{definition} \newtheorem{theorem}{Theorem}[section] \newtheorem{corollary}{Corollary}[theorem] \newtheorem{lemma}[theorem]{Lemma} \newtheorem{definition}{Definition}[section] \author{Connor Duncan} \date{\today}
\title{Physics-105-Lecture-Notes-05-02-2019}
\begin{document}
\maketitle\tableofcontents
\noindent\abstract{A single PDF with all lectures in a single document can be downloaded at \url{https://www.dropbox.com/sh/8sqzvxghvbjifco/AAC9LoSRnsRQDp7pYedgWpQMa?dl=0}. The password is 'analytic.mech.dsp'.
 This file was automatically generated using a script, so there might be some errors. If there are, you can contact me at \url{mailto:ctdunc@berkeley.edu}.}
\section{Infinitesimal Canonical Transformations} (this stuff is not examinable) Consider some \begin{equation} \tilde H=H+\pdv{F(g,Q,p,P,t}{t} \end{equation} such that $p_1=\pdv{F}{q}$, $P_1=-\pdv{F}{Q_1}$ where $F$ is some generating function. Maybe we let $S=qP$, then it's just the identity transformation. If we taek \begin{equation} S=qP+\epsilon G(q,p,t) \end{equation} then \begin{equation} \tilde H=H+\pdv{S}{t} \end{equation} and $p=\pdv{S}{q}(q,P,t)$, and $Q=\pdv{S}{P}$, so we want to fund $P(\epsilon)\approx P(\epsilon=0)+\epsilon\pdv{P}{\epsilon}\ldots$, and same for $Q$ which gives us \begin{equation} 0=\pdv{p}{\epsilon}+\pdv{G}{q} \end{equation} together, these give \begin{align} P=p-\epsilon\pdv{G}{q}\\ Q=q+\epsilon\pdv{G}{p} \end{align} Now, if we choose our $G$ to be the hamiltonian, i.e. $G(q,p,t)=H(q,p,t)$, then \begin{align} p\approx p-\epsilon\pdv{H}{q}=p+\epsilon\dot p \\ q\approx q+\epsilon\pdv{H}{p}=q+\epsilon\dot q \end{align} If we take $\epsilon$ to be $dt$, then we can think of the hamiltonian as being the propagator of time translation, i.e. using the hamiltonian as a generating function of canonical transformations gives out time translation. If we take $G=\hat z\cdot\hat L_z$, we get out that $xp_-yp_x$, which takes \begin{align} X\approx x+\epsilon\pdv{G}{p_x}=x-\epsilon y\\ Y=y+\epsilon\pdv{G}{p_y}=y+\epsilon x \end{align} which shows angular momentum is the generator of rotation!! (Shoutout to my 137a peeps) What if we want to be more general, examining what happens to a function $u(Q,P,t)$ under such transformations? \begin{equation} \left.\dv{u}{t}\right|_{t=0}=\left.\left(\pdv{u}{Q}\pdv{Q}{\epsilon}+\pdv{u}{P}\pdv{P}{\epsilon}\right)\right|_{\epsilon=0} \end{equation} which is approximately \begin{align} \pdv{u}{\epsilon}=\left(\pdv{u}{q}\pdv{G}{p}-\pdv{u}{p}\pdv{G}{q}\right) \end{align} which takes \begin{equation} u(t)=u(q,p,t)+\epsilon\left(\pdv{u}{q}\pdv{G}{p}-\pdv{u}{p}\pdv{G}{q}\right) \end{equation} where the term on the left is called the Poisson Bracket of $\{u,G\}$ \begin{equation} \{u,G\}=\pdv{u}{q}\pdv{G}{p}-\pdv{u}{p}\pdv{G}{q} \end{equation} Generally, if we take \begin{align} \dv{u}{t}=\pdv{u}{q}\pdv{q,t}+\pdv{u}{p}\pdv{p}{t}+\pdv{u}{t} \\ =\pdv{u}{q}\pdv{H}{p}-\pdv{u}{p}\pdv{H}{q}+\pdv{u}{t} \\ =\{u,H\}+\pdv{u}{t} \end{align} some more properties of poisson brackets Bale isnt' going to derive. \begin{itemize} \item $[u,v]=-[v,u]$ \item $[u,u]=0$ \item $[(u_1+u_2),v]=[u_1,v]+[u_2,v]$ \item $[u_1u_2,v]=u_1[u_2,v]+[u_1,v]u_2$ \item (also jacobi identity) \end{itemize} and if we have \begin{equation} \pdv{u}{t}=[u,H]=0 \end{equation} then $u$ is a constant of motion. Also, you can take poisson brackets and generate more conserved quantities (i.e. \begin{equation} \dv{t}[u,v]=0 \end{equation}
\end{document}
