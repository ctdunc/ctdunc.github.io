\documentclass{article} \usepackage{amsmath} \usepackage{amssymb} \usepackage{amsthm} \usepackage[margin=0.2in]{geometry} \usepackage{hyperref} \usepackage{physics} \usepackage{tikz} \usepackage{mathtools} \mathtoolsset{showonlyrefs} \theoremstyle{definition} \newtheorem{theorem}{Theorem}[section] \newtheorem{corollary}{Corollary}[theorem] \newtheorem{lemma}[theorem]{Lemma} \newtheorem{definition}{Definition}[section] \author{Connor Duncan} \date{\today}
\title{Physics-105-Lecture-Notes-04-16-2019}
\begin{document}
\maketitle\tableofcontents
\noindent\abstract{A single PDF with all lectures in a single document can be downloaded at \url{https://www.dropbox.com/sh/8sqzvxghvbjifco/AAC9LoSRnsRQDp7pYedgWpQMa?dl=0}. The password is 'analytic.mech.dsp'.
 This file was automatically generated using a script, so there might be some errors. If there are, you can contact me at \url{mailto:ctdunc@berkeley.edu}.}
\subsection{Something New} Considedr $N$ masses connected by some medium with tension $\tau$, whihcm eans we need $y_n$ $n=1\rightarrow N$, withh yeach $y_k$ given as the displacement above equilibrium Lagrangian given as \begin{equation} \mathcal L=\frac{1}{2}\left(\sum_{k=1}^Nm\dot y_k^2-\sum_{k=0}^n\frac{\tau}{d}(y_k-y_{k+1})^2\right):y_0=0,y_{N+1}=0 \end{equation} we need to assume tht $m_{ik}=m\delta_{ik}$, and we have that \begin{equation} V_{ik}=\frac{\tau}{d} \begin{bmatrix} 2 & -1 & 0 & \ldots & 0\\ -1 & 2 & -1 & 0 & \ldots\\ 0 & -1 & 2 & -1 & \ldots\\ \vdots & 0 & \ddots & \ddots & \vdots\\ 0 & \ldots & \ldots & -1 & 2 \end{bmatrix} \end{equation} So, if we want to find the eigenmodes, we just take $||\hat V-\omega^2\hat m||=0$, which gives us \begin{equation} \det\left|\begin{bmatrix} 2-\frac{\omega^2}{\omega_0^2} & -1 & \ldots & 0\\ -1 & \ddots & \ddots & \vdots\\ \vdots & \ddots & \ddots & \vdots\\ 0 & \ldots & \ldots & 2-\frac{\omega^2}{\omega_0^2} \end{bmatrix} \right| \end{equation} If we wanted to solve this for two masses, we have \begin{equation} \det\left|\begin{matrix} x & -1\\ -1 & x \end{matrix} \right| \end{equation} in 3 dimensions, we should have \begin{equation} \det\left|\begin{matrix} x & -1 & 0\\ -1 & x & -1\\ 0 & -1 & x \end{matrix} \right| \end{equation} which gives eigenfrequencies $\omega=\omega_0\sqrt{2}$ and $\omega=\omega_0\sqrt{2\pm\sqrt{2}}$. Also always true that $\omega\not{\rightarrow}\infty$. You can just use a computer to do this, but we can use recursive relations. Lets call $\Lambda_N$ the determinant of our giant matrix of said form. You can use a recurrence relation $\Lambda_N=x\Lambda_{N-1}-\Lambda_{N-2}$. If we write down lagranges equation, we get \begin{align} m\ddot y_k=\pdv{\mathcal L}{y_k}=-\frac{\tau}{d}(y_k-y_{k+1})-\frac{\tau}{d}(y_k-y_{k-1})\\ \end{align} if we guess teh form is of $y_k=e^{i(k\gamma+\delta)}$, we can reqrite the above as \begin{equation} \omega^2e^{ik\gamma}=\omega_0^2e^{ik\gamma}(1-e^{i\gamma})+\omega_0^2e^{ik\gamma}(1-e^{-i\gamma}) \end{equation} which can simplify do \begin{equation} \omega^2=\omega_0^2(2-2\cos\gamma)=4\omega_0^2\sin^2\frac{\gamma}{2} \end{equation} of course $\gamma$ not arbitrary, because we must have $y_0=y_{N+1}=0$, which means we must have \begin{align} y_k=\cos(ik\gamma+\delta)\cos(\omega t+\varphi) \end{align} which givess $y_0=\cos\delta\cos(\omega t+\varphi)\Rightarrow\delta=\frac{(2n+1)\pi}{2}$. and $\gamma=\frac{\pi n}{N+1}$, which just gives out standing waves in the end! We can think of $\gamma/d$ as the wavenumber, which, if we think about $\frac{2\pi}{\lambda}=\frac{\gamma}{d}$, our wavenumber is not arbitrary, which means we need to have that equal $\frac{\pi n}{d(N+1)}$, where the denominator is $L$, the length between boundaries of the medium. Or, $\frac{L}{\lambda}=\frac{n}{2}$. \subsection{Traveling Wave} This is \emph{not a general solution} however. Imagine the case where we perturb one mass in the center of hundreds of masses. $\gamma$ is not fixed here, because the boundary conditions don't know about the pertubation until later. We can try to find a solution for some traveling wave $y_k=A_ke^{i(k\gamma-\omega t)}$, where there's some dispersion relation $\omega=2\omega_0\sin\frac{\gamma}{2}$. If we assume small $\gamma$, we have immediately that $\omega=\omega_0\gamma=\omega_0d\vec{k}$ where $\vec{k}$ is the wavevector, with some speed of sound $c_S=\sqrt{\frac{\tau}{md}}\times d=\sqrt{\frac{\tau d}{m}}$. We can be more strict though. Let's consider the continuous limit of our system, letting $d\rightarrow 0,k\rightarrow\infty, m\rightarrow 0,$ with linear mass density $\rho=\frac{m}{d}$. Now, we have some function $y(x,t)$, with \begin{equation} \ddot y=\omega_0^2\pdv{y_{k+1/2}}{x}d-\omega_0^2\pdv{y_{k-1/2}}{x}d=\omega_0^2d^2\pdv[2]{y}{x} \end{equation} which is just the wave equation, for constant density, linear mass density. \begin{equation} \pdv[2]{y}{t}=\omega_0^2d^2\pdv[2]{y}{x} \end{equation} Going through the derivation again, we have the more general form of \begin{equation} \pdv[2]{y}{t}=\frac{1}{\rho(x)}\pdv[]{x}\left[\tau(x)\pdv{y}{x}\right] \end{equation} If we rewrite the wave equation as \begin{equation} \pdv[2]{y}{t}=c_s^2\pdv[2]{y}{x} \end{equation} we find that in this wave equation, $\frac{\omega}{c_s}L=\pi n$, which is just a large limit of the formula we had before. \subsection{Lagrangian Density} We can also do this using the lagrangian, by making an argument \begin{equation} \mathcal L=\frac{1}{2}\rho(x)\left(\pdv{y}{t}\right)^2-\frac{1}{2}\tau(x)\left(\pdv{y}{x}\right)^2 \end{equation} The way we arrive at this conclusion is by varying the functional \begin{equation} \delta\int_{t_1}^{t_2}\int_\mathcal{D}\mathcal L(x,t,y,\partial_xy,\partial_ty)=0 \end{equation} A WHOLE LOT OF ALGEBRA LATER, the correct answer will fall outof the thing. the final form, we get \begin{equation} \pdv{t}\left(\pdv{\mathcal L}{(\partial_t y)}\right)+\pdv{x}\left(\pdv{\mathcal L}{(\partial_xy)}\right)=\pdv{\mathcal L}{y} \end{equation} \subsection{Hamiltonian Density} We have some $H$, we can introduct some continuous medium version of this, with $\mathcal L(x,t,y,\partial_ty,\partial_xy)$, with some generalized momenta $\vec{p}=\pdv{\mathcal L}{(\partial_ty)}$, then introduce hamiltonain denstiy \begin{equation} \mathcal H=p\partial_ty=\mathcal L=\frac{1}{2}\rho(x)\left(\pdv{y}{t}\right)^2+\frac{1}{2}\tau(x)\left(\pdv{y}{x}\right)^2 \end{equation}
\end{document}
