\documentclass{article} 
\usepackage{amsmath} 
\usepackage{amssymb} 
\usepackage{amsthm} 
\usepackage[margin=0.2in]{geometry} 
\usepackage{hyperref} 
\usepackage{physics} 
\usepackage{tikz} 
\usepackage{mathtools}
\mathtoolsset{showonlyrefs} 
\theoremstyle{definition} 
\newtheorem{theorem}{Theorem}[section] 
\newtheorem{corollary}{Corollary}[theorem] 
\newtheorem{lemma}[theorem]{Lemma} 
\newtheorem{definition}{Definition}[section] 

\author{Connor Duncan}
\date{\today}

\title{notes-09-23-2019}
\begin{document}
\abstract{A single document copy of these notes, as well as a mirror of every note, can be found at \url{connorduncan.xyz/notes}}
We want to mix $\ket{k},\ket{-k}$ at $\alpha\ket{k}+\beta\ket{-k}$, which comes down to solve the schroedinger equaltion in this degenerate subspace. This gives \begin{equation} \begin{bmatrix} \bra{k}H\ket{k} & \bra{k}H\ket{k'}\\ \bra{k'}H\ket{k} & \bra{k'}H\ket{k'} \end{bmatrix} \begin{bmatrix} \alpha \\ \beta \end{bmatrix} =E \begin{bmatrix} \alpha \\ \beta \end{bmatrix} \end{equation} This becomes \begin{equation} \begin{bmatrix} E_0(k)+V_0 & V_n\\ V_n^* & E_0(k')+V_0 \end{bmatrix} = \begin{bmatrix} \alpha \\ \beta \end{bmatrix} =E \begin{bmatrix} \alpha \\ \beta \end{bmatrix} \end{equation} We know that $E_0(k)=E_0(k'=-k)=\frac{n^2\hbar^2\pi^2}{2ma^2}$, with after a bit of algebra gives us mixing \begin{equation} E=\frac{\hbar^2}{2m}\frac{n^2\pi^2}{a^2}+V_0\pm|V_n| \end{equation}  \paragraph{Case 3: Close to edge of BZ} We want to take a ``continuous'' approach to the edge at each band. We're gonna take $k=\frac{n\pi}{a}+\delta$, and $k'=-\frac{n\pi}{a}+\delta'$. This is going to give us some state where $\exists$ matrix elements, but the gap between energy levels no longer be zero. Now, we want to solve the eigenvalue equaitons for \begin{equation} E_\pm=\frac{\hbar^2}{2m}\left(\frac{n^2\pi^2}{a^2}\pm\delta^2\right)+V_0\pm\sqrt{|V_n|^2+\left(\frac{\hbar^2}{2m}\frac{2\pi n}{a}\delta\right)^2} \end{equation} In the limit $\delta\rightarrow 0$, we justt get our original answer back, which is good. We can check out a few limits. First, let $\delta\gg V_n$. To first order, it becomes \begin{equation} E_\pm=E_0\left(\frac{n\pi}{a}\pm\delta\right)+V_0\pm\frac{|V_n|^2}{E\left(\frac{n\pi}{a}+\delta\right)-E\left(\frac{n\pi}{a}-\delta\right)} \end{equation} Limit 2 is for $\delta\ll V_n$, \begin{equation} E_\pi=\frac{\hbar^2}{2m}\frac{n^2\pi^2}{a^2}+V_0\pm|V_n|+\frac{\hbar^2}{2m}\left(1+\frac{1}{|V_n|}\frac{n^2\hbar^2\pi^2}{ma^2}\right)^2 \end{equation} There are some things we should take away from this \begin{enumerate} \item for $k\ll\frac{\pi}{a}$, we have basically a free electron \item for $k=\frac{\pi n}{a}$, the spectrum splits as $2|V_n|$. \end{enumerate} \section{Variational Method} In most cases, it's basically impossible to solve schroedingers equation, because we can't diagonalize that ``big ass matrix''\footnote{Norman Yao, circa 2019.}. The idea is to find the ground state of some hamiltonian $H$. Then, we make the staement \begin{equation} \forall\ket{\psi}; E(\psi)=\frac{\bra{\psi}H\ket{\psi}}{\bra{\psi}\ket{\psi}}\geq E_0 \end{equation} i.e. any state's energy will be larger than the ground state energy. We can find some trial wavefunctions parameterized by $\{\alpha,\beta,\gamma\dots\}$. Then, the goal becoems to minimize $E(\psi)$ with respect to these variational parameters which results in \textbf{a rigorous bound on the ground state energy}. In principle, we can always improve our bound by adding more variational parameters. \subsection{Ex: Free Particle} Let's take \begin{equation} H_0=\frac{p^2}{2m}+\lambda x^4 \end{equation} The trial wavefunctions we are going to use are $\psi(x,\alpha)=\left(\frac{\alpha}{\pi}\right)^{1/4}e^{-\frac{1}{2}\alpha x^2}$. We choose our functions to be ``nice'' based on the hamiltonian. Prof. Yao wrote this on the board \begin{enumerate} \item Gaussians are easy for this $H$ \item Symmetric Function w/ nodes \end{enumerate} We then calculate the variational energy \begin{equation} E(\alpha)=\sqrt{\frac{\alpha}{\pi}}\int dx e^{-\frac{\alpha x^2}{2}}\left( -\frac{\hbar^2 }{2m}\pdv[2]{x}+\lambda x^4 \right)e^{-\frac{\alpha x^2}{2}} =\frac{\hbar^2}{2m}\alpha+\frac{3\lambda}{4\alpha^2} \end{equation} The latter term means that we have $\alpha\rightarrow\infty$ wants localized wavefunction to be minimal, and the former is minimzed by $\alpha\rightarrow 0$, saying that low momentum minimizes that component of the wf. We can solve this for \begin{align} \alpha_0=\left(\frac{6m\lambda}{\hbar^2}\right)^{1/3} && E(\alpha_0)=\frac{3}{8}\left(\frac{6\hbar^4\lambda}{m^2}\right)^{1/3} \end{align} <++>
\end{document}
