\documentclass{article} 
\usepackage{amsmath} 
\usepackage{amssymb} 
\usepackage{amsthm} 
\usepackage[margin=0.2in]{geometry} 
\usepackage{hyperref} 
\usepackage{physics} 
\usepackage{tikz} 
\usepackage{mathtools}
\usepackage{graphicx}\graphicspath{{./images/}}
\mathtoolsset{showonlyrefs} 
\theoremstyle{definition} 
\newtheorem{theorem}{Theorem}[section] 
\newtheorem{corollary}{Corollary}[theorem] 
\newtheorem{lemma}[theorem]{Lemma} 
\newtheorem{definition}{Definition}[section] 

\author{Connor Duncan}
\date{\today}

\title{notes-11-01-2019}
\begin{document}
\abstract{A single document copy of these notes, as well as a mirror of every note, can be found at \url{connorduncan.xyz/notes}}
\subsection{Formalism: First Order} \rule{\textwidth}{1pt} \subsubsection{Infinite Time Perturbation} Recall we have some $H=H_0+H_1(t)$ where $H_1=g(t)\hat V$. We defined $d_f(t)=c_f(t)e^{iE_f^{(0)}t}$, in order to undo the time dependence of $c_f$ in an eigenstate, where we just shift the phase oscillation in the other direction. We computed such $d_f(t)$ to be as follows: \begin{equation} d_f(t)=\delta_{if}-\frac{i}{\hbar}\int_{t_0}^{t_f}\bra{f^{(0)}}H_1(t')\ket{i^{(0)}}e^{i\omega_{fi}t'}dt' =\delta_{if}-\frac{i}{\hbar}\bra{f^{(0)}}\hat V\ket{i^{(0)}}\int_{t_0}^tdt'g(t')e^{i\omega_{if}t'} \end{equation} In the limit where $t_0\rightarrow-\infty$, $t\rightarrow\infty$, we have that this expression is just a fourier transform of $g(t)$, and can reexpress this as \begin{equation} d_f(t)=\delta_{if}-\frac{i\tilde g(t)}{\hbar}\bra{f^{(0)}}\hat V\ket{i^{(0)}} \end{equation} where $\tilde g$ represents the fourier transform of $g$. \subsubsection{Sudden Perturbation} If we have \begin{equation} H_1(t)=\Theta(t)\hat V \end{equation} It sets our lower limit of integration to 0, and so we're left with \begin{equation} d_f^{(1)}(t)=-\frac{i}{\hbar}\bra{f^{(0)}}\hat V\ket{i^{(0)}}\int_0^tdt'e^{i\omega_{fi}t} = \left\{\begin{matrix*}[l] -\frac{2i\bra{f}\hat V\ket{i}}{\hbar\omega_{fi}}\sin(\frac{\omega_{fi}t}{2})e^{i\omega_{fi}t/2} & f\neq i \\ -\frac{i}{\hbar}\bra{i}\hat V\ket{i} & f=i \end{matrix*}\right. \end{equation} This is actually a really nice sanity check, since $\bra{i^{(0)}}\hat V\ket{i^{(0)}}$ is the energy we'd expect from first order perturbation theory without time dependence! So our phase shift is nicely consistent with our previous results. If we want the transition probability, we can just take \begin{equation} p_{i\rightarrow f}=\frac{2}{\hbar^2}\left|\bra{f^{(0)}}\hat V\ket{i^{(0)}}\right|^2\frac{2\sin^2(\omega_{fi}t/2)}{\omega_fi t}t \end{equation} where we call $F(\omega_{fi},t)$, so we have \begin{equation} p_{i\rightarrow f}=\frac{2}{\hbar^2}\left|\bra{f^{(0)}}\hat V\ket{i^{(0)}}\right|^2\frac{2\sin^2(\omega_{fi}t/2)}{F(\omega_{if},t)}t \end{equation} \subsection{Examples} \subsubsection{Harmonic Oscillator (infinite time)} Consider \begin{align} H_0=\frac{p^2}{2m}+\frac{1}{2}m\omega^2 x^2 \\ H_1(t)=-e\varepsilon\hat xe^{-t^2/\tau^2} \end{align} We want to know what the amplitude $d_n(\infty)$ is, the probability to find the oscillator in state $n$ after finite time, assuming that we start in the ground state? We just put it in to the formula, \begin{equation} d_n(t)=-\frac{i}{\hbar}-e\varepsilon\bra{n}\hat x\ket{0}\int dt e^{-t^2/\tau^2}e^{in\omega t} \end{equation} We can just take \begin{equation} d_n(t)=\frac{ie\varepsilon}{\hbar}\sqrt{\frac{\hbar}{2m\omega}}\bra{n}(a_-+a_+)\ket{0}\int dte^{-t^2/\tau^2-in\omega t} =\delta_{n1}\frac{ie\varepsilon}{\hbar}\sqrt{\frac{\hbar}{2m\omega}}\sqrt{\pi\tau^2}e^{-\omega^2\tau^2/4} \end{equation} which gives \begin{equation} |d_n|^2=\delta_{10}\frac{e^2\varepsilon^2\pi\tau^2}{2m\omega\hbar}e^{-(\omega\tau)^2/2} \end{equation}
\end{document}
