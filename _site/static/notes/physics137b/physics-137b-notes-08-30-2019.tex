\documentclass{article} 
\usepackage{amsmath} 
\usepackage{amssymb} 
\usepackage{amsthm} 
\usepackage[margin=0.2in]{geometry} 
\usepackage{hyperref} 
\usepackage{physics} 
\usepackage{tikz} 
\usepackage{mathtools}
\mathtoolsset{showonlyrefs} 
\theoremstyle{definition} 
\newtheorem{theorem}{Theorem}[section] 
\newtheorem{corollary}{Corollary}[theorem] 
\newtheorem{lemma}[theorem]{Lemma} 
\newtheorem{definition}{Definition}[section] 

\author{Connor Duncan}
\date{\today}
\title{notes-08-30-2019}
\begin{document}
\subsection{Operators$\rightarrow$ observables}
\subsubsection{Operators, Bases, Linear Algebra}
Observables include our old friends $\hat x,\hat p,\hat H, \hat L\,\hat s$.

What is an operator tho? 
\begin{equation}
	\begin{matrix}
		\hat A:		&	\mathcal H\rightarrow\mathcal H
	\end{matrix}
\end{equation}
also $\exists$ adjoint operator, $A^{\dag}$ from $\mathcal H$ to itself.

We can construct projection operators by taking the outer product of two vectors.

Also lets us have some resolution of identity by summing over the projection operators in any orthonormal basis.

Also, matrix representations exist by arranging $A_{mn}=\bra{\varphi_m}\hat A\ket{\varphi_n}$, for some particular basis $\varphi_i$

In other words
\begin{equation}
	\hat A=\sum_{ij}\ket{\varphi_i}A_{ij}\bra{\varphi_j}
\end{equation}

Cute stuff with the momentum operator.

\end{document}
