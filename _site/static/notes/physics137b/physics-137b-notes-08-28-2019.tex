\documentclass{article} 
\usepackage{amsmath} 
\usepackage{amssymb} 
\usepackage{amsthm} 
\usepackage[margin=0.2in]{geometry} 
\usepackage{hyperref} 
\usepackage{physics} 
\usepackage{tikz} 
\usepackage{mathtools}
\mathtoolsset{showonlyrefs} 
\theoremstyle{definition} 
\newtheorem{theorem}{Theorem}[section] 
\newtheorem{corollary}{Corollary}[theorem] 
\newtheorem{lemma}[theorem]{Lemma} 
\newtheorem{definition}{Definition}[section] 

\author{Connor Duncan}
\date{\today}
\title{notes-09-04-2019}

\begin{document}
\section*{Administration}
30\% Homework, 30\% Midterm (Tentative Midterm Oct. 18), 40\% Final. 

Quizzes in section (monday), don't count towards the final grade.

The syllabus will track basically as follows:
\begin{enumerate}
	\item Stationary Perturbation Theory (time independent)
	\item Variational Principle (not in townsend)
	\item WKB (semicalssical) approximation (not in townsend, may skip)
	\item Time Dependent Perturbation Theory
	\item Coupling of Quantum Particles + Electromagnetic Fields (Aharonov-Bohm effect)
	\item Quantize Light, Physics of Photons and Photon-Atom interaction.
\end{enumerate}

\section{Review}

\subsection{State Vector, Kets, Bras}
$\ket{\psi}, \psi(x)$, difference between the two. 
State vector $\ket{\psi}$ can be constructed out of other vectors. I.e. $\ket{\psi}=\sum_{i}c_i\ket{\psi_i}$.

Our hilbert space is over $\mathbb{C}$.

Bra objects are the dual of their respective kets.
\begin{align}
	\ket{\psi}\in\mathcal{H}	&&	\bra{\psi}\in\mathcal{H}^*
\end{align}
$\mathcal{H}^*$ is the space of linear operators
\begin{equation}
		\begin{matrix}
			\bra{\psi}:	&	\mathcal{H}\rightarrow\mathbb{C}\\
					&	\bra{\psi}\ket{\psi}=\bra{\psi}\ket{\psi}^*=||\ket{\psi}||^2
		\end{matrix}
\end{equation}
We have to make a choice of basis in order to gain any useful information from this vector space.

A basis is any minimal collection of vectors whose span is the desired hilbert space.

Typically, we choose an orthonormal basis such that 
$\bra{\psi_i}\ket{\psi_j}=\delta_{ij}$

Under this expression, we have
\begin{align}
	\ket{\psi}=\sum_i\psi_i\ket{\phi_i}\\
	\bra{\psi}\ket{\psi}\sum_i\sum_j\psi_i^*\psi_j\bra{\phi_i}\ket{\phi_j}=\sum_i|\phi_i|^2=1
\end{align}

Inner product is defined in the usual way.

We are allowed to have continuous bases. E.g. a particle on a 1-d line.
We can attempt to measure a position of the particle on the line. We essentially use the same formalism, but with
\begin{equation}
	\bra{x}\ket{x'}=\delta(x-x')
\end{equation}
where $\delta$ is now the Dirac Delta function.

Similarly, we have
\begin{align}
	\ket{\psi}=\int dx\psi(x)\ket{x}
	&&
	\bra{\psi}=\int dx\psi^*(x)\bra{x}
\end{align}
Leading to the usual definition of the probability of $\ket{\psi}$ at a given $x$ as
\begin{equation}
	\bra{x}\ket{\psi}=\int dx'\psi(x')\bra{x}\ket{x'}=\psi(x)
\end{equation}
and our other favorite
\begin{equation}
	\bra{\chi}\ket{\psi}=\int dx\chi^*(x)\psi(x)
\end{equation}
\end{document}
